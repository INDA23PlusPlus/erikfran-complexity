\documentclass{article}

\usepackage{polynom}
\usepackage{amsmath}
\usepackage{tkz-tab}
\usepackage{xpatch}
\usepackage{amsfonts}
\usepackage{hyperref}
\usepackage{amssymb}
    \usepackage{fancyhdr}
    \pagestyle{fancy}
    \rhead{Erik Frankling}

\usepackage{mathtools}


\DeclarePairedDelimiter\ceil{\lceil}{\rceil}
\DeclarePairedDelimiter\floor{\lfloor}{\rfloor}

\title{Komplexitet och Korrekhet}
\author{Erik Frankling}
\begin{document}

\section{Induktion}
\subsection{Uppgift 1}
Bevisa med induktion att 
\[ \sum_{i=1}^{n} i ^ 2 = \frac{n (n + 1) (2n + 1)}{6} \] 
för alla $n \ge 1$ där $n \in \mathbb{Z}$\\


\textit{Basfall:} $n = 1$
\[ P(1) VL = \sum_{i=1}^{1} i ^ 2  = 1 ^ 2 = \frac{6}{6} = \frac{1 (1 + 1) (2 * 1 + 1)}{6} = HL \]

\textit{Induktionsantagande:} Antag att $P(k)$ är sann vilket ger att 
\[ \sum_{i=1}^{k} i ^ 2 = \frac{k (k + 1) (2k + 1)}{6} \]

\textit{Induktionssteget:}
Vi ska visa att givet induktionsantagandet är $P(k + 1)$ sant.
\[ VL = \sum_{i=1}^{k + 1} i ^ 2 = (k + 1) ^ 2 + \sum_{i=1}^{k} i ^ 2 \]
Nu stoppar vi in det vi kom fram till i induktionsantagandet och får
\[ VL = (k + 1) ^ 2 + \frac{k (k + 1) (2k + 1)}{6} = \]
\[ = \frac{ 6 (k ^ 2 + 2k + 1) + 2k ^ 3 + 3k ^ 2 + k}{6} = \]
\[ = \frac{ 2k ^ 3 + 9k ^ 2 + 13k + 6 }{6}\]
\[ HL = \frac{(k + 1) (k + 2) (2 (k + 1) + 1)}{6} = \]
\[ = \frac{(k ^ 2 + 3k + 2) (2k + 3)}{6} = \]
\[ = \frac{2k ^ 3 + 9k ^ 2 + 13k + 6}{6} = VL \]
Påståendet följer enligt induktionsprincipen.

\subsection{Uppgift 2}
Bevisa med induktion att 
\[ \sum_{j=1}^{n} (2j - 1) = n ^ 2 \]
för alla $n \ge 1$ där $n \in \mathbb{Z}$\\


\textit{Basfall:} $n = 1$
\[ P(1) VL = \sum_{j=1}^{1} (2j - 1)  = 2 * 1 - 1 = 1 = 1 ^ 2 = HL \]

\textit{Induktionsantagande:} Antag att $P(k)$ är sann vilket ger att 
\[ \sum_{j=1}^{k} (2j - 1) = k ^ 2 \]

\textit{Induktionssteget:}
Vi ska visa att givet induktionsantagandet är $P(k + 1)$ sant.
\[ VL = \sum_{j=1}^{1} (2j - 1) = 2 (k + 1) - 1 + \sum_{i=1}^{k} (2j - 1) \]
Nu stoppar vi in det vi kom fram till i induktionsantagandet och får
\[ VL = 2 (k + 1) - 1 + k ^ 2 = \]
\[ = k ^ 2 + 2k + 1 = \]
\[ = (k + 1) ^ 2 = HL \]
Påståendet följer enligt induktionsprincipen.
\section{Iterativ korrekhet}

Innan loopen har vi att $res_0 = 1.0 = x^0$

I slutet av iteration $i$ har vi att $res_{i + 1} = x * res_i = $ {induktionssteg} $ = res_i = x ^ i$

Den sista iterationen är $i = n - 1$ då vi startar från i = 0 vilket ger att $res_{n - 1 + 1} = res_n = x ^ {n}$

Vi har då visat att $f(x, n) = x ^ n$.

Antalet iterationer är n vilket ger att tidskomplexiteten är $O(n)$.

\section{Rekursiv korrekhet}
\begin{equation*}
expRecursive(x, n) = f(x, n) = 
  \begin{cases}
    g(x ,n) \quad \text{då } 0 \le n \le 4\\
    f(x, \floor*{\frac{n}{2}}) \cdot \floor*{\frac{n + 1}{2}}) \quad \text{då } 0 \le n \le 4\\
  \end{cases}
\end{equation*}
\[ expIterative(x, n) = g(x, n) = x ^ n \]
där $n \in \mathbb{Z}$
Påstående $P(n)$:
\[ P(n) \Longleftrightarrow expRecursive(x, n) = x ^ n \]
\textit{Basfall:}
\[ f(x, n) = g(x, n) = x ^ n \longleftarrow P(n), 0 \le n \le 4 \]
\textit{Induktionssteg:} Antag $P(n)$ för alla $ 4 < n < m$

Då m är på formen $m = 2k$
\[ g(x, m) = g(x, k) \cdot g(x, k) = \text{induktionsantagande} = x ^ k \cdot x ^ k = x ^ {2k} = x ^ m \]

Då m är på formen $m = 2k + 1$
\[ g(x, m) = g(x, k) \cdot g(x, k + 1) = \text{induktionsantagande} = x ^ k \cdot x ^ {k + 1} = x ^ {2k + 1} = x ^ m \]

Påståendet $P(n)$ följer då enligt starka induktionsprincipen för alla $n > 0$ där $n \in \mathbb{Z}$

Rekursiv tidskomplexitet ges av formeln
\[ T(n) = aT(n/b) + f(n) \]
där $a = 2, b = 2$ och $f(n) \in \theta(1)$ mästarsatsen ger då
\[ T(n) \in \Theta(n ^ {log_22}) = \Theta(n) \]
\end{document}